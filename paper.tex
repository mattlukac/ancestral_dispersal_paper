\documentclass{article}



\usepackage{arxiv}

\usepackage[utf8]{inputenc} % allow utf-8 input
\usepackage[T1]{fontenc}    % use 8-bit T1 fonts
\usepackage{hyperref}       % hyperlinks
\usepackage{url}            % simple URL typesetting
\usepackage{booktabs}       % professional-quality tables
\usepackage{amsfonts}       % blackboard math symbols
\usepackage{nicefrac}       % compact symbols for 1/2, etc.
\usepackage{microtype}      % microtypography
\usepackage{lipsum}		% Can be removed after putting your text content
\usepackage{graphicx}
\usepackage{natbib}
\usepackage{doi}



\title{Inferring Maps of Ancestral Dispersal}

%\date{September 9, 1985}	% Here you can change the date presented in the paper title
%\date{} 					% Or removing it

\author{{Matt Lukac} \\
	Institute for Ecology and Evolution\\
	Department of Biology\\
	University of Oregon\\
	\texttt{mlukac@uoregon.edu} \\
	%% examples of more authors
	\And
	{Peter Ralph} \\
	Institute for Ecology and Evolution\\
	Department of Biology\\
	University of Oregon\\
	\texttt{plr@uoregon.edu} \\
	%% \AND
	%% Coauthor \\
	%% Affiliation \\
	%% Address \\
	%% \texttt{email} \\
	%% \And
	%% Coauthor \\
	%% Affiliation \\
	%% Address \\
	%% \texttt{email} \\
	%% \And
	%% Coauthor \\
	%% Affiliation \\
	%% Address \\
	%% \texttt{email} \\
}

% Uncomment to remove the date
%\date{}

% Uncomment to override  the `A preprint' in the header
%\renewcommand{\headeright}{Preprint}
%\renewcommand{\undertitle}{Preprint}
\renewcommand{\shorttitle}{Ancestral Dispersal}

%%% Add PDF metadata to help others organize their library
%%% Once the PDF is generated, you can check the metadata with
%%% $ pdfinfo template.pdf
\hypersetup{
	pdftitle={ancestral dispersal},
	pdfsubject={popgen},
	pdfauthor={Matt Lukac},
	pdfkeywords={coalescent, hitting times, diffusion},
}

\begin{document}
	\maketitle
	
	\begin{abstract}
		Patterns of isolation by distance (IBD) can be shaped by
		the movements of a population's genetic ancestors.
		Here, we model pairwise mean coalescent time as a function of space
		and use spatiogenetic data to approximate the reverse time
		dispersal of lineages through space.
		We use an advection-diffusion model for the spatial coalescent process
		and efficiently compute gradients from the adjoint state equation.
		Model fit is assessed with a ground-truth parameter values, 
		as well as forward time spatial simulations.
		We then fit the model to North American ground beetle data
		and discuss implications.
	\end{abstract}
	
	
	% keywords can be removed
	\keywords{coalescent \and hitting times \and diffusion \and landscape genetics \and constrained optimization}
	
	
	\section{Introduction}
	\begin{itemize}
		\item describe forwards process, dispersal, mu, sigma
		\item why do mu, sigma vary with space?
		\item contrast IBD plot to spatial dxy heatmap
		\item in reality mu != 0 and sigma is not constant
		\item other spatial inference methods (MAPS, EEMS, circuitscape)
		\item other sigma inference methods using Wright-Malecot (ask Chris)
		\item inference target is in infinite dimensional function space (stands out from existing methods)
	\end{itemize}
	
	\section{Methods}
	\begin{itemize}
		\item Wright-Malecot as solution to 1D landscape problem
		\item talk about reverse time process
		\item segue to general PDE
		\item explain loss function, constrained optimization
		\item Lagrangian?
		\item adjoint method takes O(1) solves, direct method takes O(N) solves
		\item fenics, finite elements (arbitrary meshes, has function space framework)
		\item hot to apply to empirical data
		\item validation slimulations
		\item mesh convergence, gradient validation
	\end{itemize}

	\section{Results}
	\begin{itemize}
		\item fit to slim sims 
		\item figs with synthetic and slim data fits
		\item empirical data (beetles, other?)
	\end{itemize}
	
	\section{Discussion}
	\begin{itemize}
		\item implications and thoughts on sampling design
		\item uncertainty
		\item dxy along genome, or locus specific
		\item interpret reverse time results to forward time
		\item non steady-state model
	\end{itemize}
	
	%\paragraph{Paragraph}
	
	\section{Examples of citations, figures, tables, references}
	\label{sec:others}
	
	\subsection{Citations}
	Citations use \verb+natbib+. The documentation may be found at
	\begin{center}
		\url{http://mirrors.ctan.org/macros/latex/contrib/natbib/natnotes.pdf}
	\end{center}
	
	Here is an example usage of the two main commands (\verb+citet+ and \verb+citep+): Some people thought a thing \citep{kour2014real, hadash2018estimate} but other people thought something else \citep{kour2014fast}. Many people have speculated that if we knew exactly why \citet{kour2014fast} thought this\dots
	
	\subsection{Figures}
	See Figure \ref{fig:fig1}. Here is how you add footnotes. \footnote{Sample of the first footnote.}
	
	\begin{figure}
		\centering
		\fbox{\rule[-.5cm]{4cm}{4cm} \rule[-.5cm]{4cm}{0cm}}
		\caption{Sample figure caption.}
		\label{fig:fig1}
	\end{figure}
	
	\subsection{Tables}
	See awesome Table~\ref{tab:table}.
	
	The documentation for \verb+booktabs+ (`Publication quality tables in LaTeX') is available from:
	\begin{center}
		\url{https://www.ctan.org/pkg/booktabs}
	\end{center}
	
	
	\begin{table}
		\caption{Sample table title}
		\centering
		\begin{tabular}{lll}
			\toprule
			\multicolumn{2}{c}{Part}                   \\
			\cmidrule(r){1-2}
			Name     & Description     & Size ($\mu$m) \\
			\midrule
			Dendrite & Input terminal  & $\sim$100     \\
			Axon     & Output terminal & $\sim$10      \\
			Soma     & Cell body       & up to $10^6$  \\
			\bottomrule
		\end{tabular}
		\label{tab:table}
	\end{table}
	
	
	\bibliographystyle{unsrtnat}
	\bibliography{references}  %%% Uncomment this line and comment out the ``thebibliography'' section below to use the external .bib file (using bibtex) .
	
	
	%%% Uncomment this section and comment out the \bibliography{references} line above to use inline references.
	% \begin{thebibliography}{1}
		
		% 	\bibitem{kour2014real}
		% 	George Kour and Raid Saabne.
		% 	\newblock Real-time segmentation of on-line handwritten arabic script.
		% 	\newblock In {\em Frontiers in Handwriting Recognition (ICFHR), 2014 14th
			% 			International Conference on}, pages 417--422. IEEE, 2014.
		
		% 	\bibitem{kour2014fast}
		% 	George Kour and Raid Saabne.
		% 	\newblock Fast classification of handwritten on-line arabic characters.
		% 	\newblock In {\em Soft Computing and Pattern Recognition (SoCPaR), 2014 6th
			% 			International Conference of}, pages 312--318. IEEE, 2014.
		
		% 	\bibitem{hadash2018estimate}
		% 	Guy Hadash, Einat Kermany, Boaz Carmeli, Ofer Lavi, George Kour, and Alon
		% 	Jacovi.
		% 	\newblock Estimate and replace: A novel approach to integrating deep neural
		% 	networks with existing applications.
		% 	\newblock {\em arXiv preprint arXiv:1804.09028}, 2018.
		
		% \end{thebibliography}
	
	
\end{document}