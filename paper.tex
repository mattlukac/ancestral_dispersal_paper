\documentclass{article}



\usepackage{arxiv}

\usepackage[utf8]{inputenc} % allow utf-8 input
\usepackage[T1]{fontenc}    % use 8-bit T1 fonts
\usepackage{hyperref}       % hyperlinks
\usepackage{url}            % simple URL typesetting
\usepackage{booktabs}       % professional-quality tables
\usepackage{amsfonts}       % blackboard math symbols
\usepackage{nicefrac}       % compact symbols for 1/2, etc.
\usepackage{microtype}      % microtypography
\usepackage{lipsum}		% Can be removed after putting your text content
\usepackage{graphicx}
\usepackage{natbib}
\usepackage{doi}



\title{Inferring Maps of Ancestral Dispersal}

%\date{September 9, 1985}	% Here you can change the date presented in the paper title
%\date{} 					% Or removing it

\author{{Matt Lukac} \\
Institute for Ecology and Evolution\\
Department of Biology\\
University of Oregon\\
\texttt{mlukac@uoregon.edu} \\
%% examples of more authors
\And
{Peter Ralph} \\
Institute for Ecology and Evolution\\
Department of Biology\\
University of Oregon\\
\texttt{plr@uoregon.edu} \\
%% \AND
%% Coauthor \\
%% Affiliation \\
%% Address \\
%% \texttt{email} \\
%% \And
%% Coauthor \\
%% Affiliation \\
%% Address \\
%% \texttt{email} \\
%% \And
%% Coauthor \\
%% Affiliation \\
%% Address \\
%% \texttt{email} \\
}

% Uncomment to remove the date
%\date{}

% Uncomment to override  the `A preprint' in the header
%\renewcommand{\headeright}{Preprint}
%\renewcommand{\undertitle}{Preprint}
\renewcommand{\shorttitle}{Ancestral Dispersal}

%%% Add PDF metadata to help others organize their library
%%% Once the PDF is generated, you can check the metadata with
%%% $ pdfinfo template.pdf
\hypersetup{
pdftitle={ancestral dispersal},
pdfsubject={popgen},
pdfauthor={Matt Lukac},
pdfkeywords={coalescent, hitting times, diffusion},
}

\begin{document}
\maketitle

\begin{abstract}
	Patterns of isolation by distance (IBD) can be shaped by
	the movements of a population's genetic ancestors.
	Here, we model pairwise mean coalescent time as a function of space
	and use spatiogenetic data to approximate the reverse time
	dispersal of lineages through space.
	We use an advection-diffusion model for the spatial coalescent process
	and efficiently compute gradients from the adjoint state equation.
	Model fit is assessed with a ground-truth parameter values,
	as well as forward time spatial simulations.
	We then fit the model to North American ground beetle data
	and discuss implications.
\end{abstract}


% keywords can be removed
\keywords{coalescent \and hitting times \and diffusion \and landscape genetics \and constrained optimization}


%%%%%%%%%%%%%%%%%%%%%%
\section{Introduction}

% \begin{itemize}
%     \item describe forwards process, dispersal, mu, sigma
%     \item why do mu, sigma vary with space?
%     \item contrast IBD plot to spatial dxy heatmap
%     \item in reality mu != 0 and sigma is not constant
%     \item other spatial inference methods (MAPS, EEMS, circuitscape)
%     \item other sigma inference methods using Wright-Malecot (ask Chris)
%     \item inference target is in infinite dimensional function space (stands out from existing methods)
% \end{itemize}

The way in which organisms disperse across a landscape
forms a fundamental part of a species' demography.
Do offspring live near to their parents or far?
Where do the parents tend to live relative to each other?
Is there preferential movement in particular directions?
These questions are as relevant to plants as to the more motile organisms
that often carry the plants' seeds or pollen.
We may be interested in the answers to understand
sharing of genetic diversity
or to predict population persistance
(a species living in a remnant fragment of good habitat
is in trouble if most offspring leave the fragment)
Of course, the most conceptually straightforward way to understand organism movement
is to directly observe it.
However, this is less straightforward in practice:
even when it is feasible to directly observe movement or parentage,
it can require a tremendous amount of effort to reach statistically useful sample sizes.

To complement direct observation of natural history,
we might then turn to population genetic data.
Since two individuals are nearly always related through a huge number of distinct
paths through the species' pedigree,
even a handful of genomes potentially provides information, albeit diffuse,
about a large number of parent-offspring relationships.
The field has long used patterns of genetic diversity to estimate various quantities
by fitting mechanistic models of discrete populations.
It is our goal in this paper to do the same, for continuous space.

Organisms live across real landscapes, and for most species
the realities of spatial habitat heterogeneity and sheer geographical separation
are very important factors in the dynamics of their populations.
We are concerned here with those quantities that determine
how genetic relatedness assorts across geography:
the displacement between parent and child, i.e., ``dispersal''.
(The displacement to \emph{which} parent?
It depends -- in the case of autosomally inherited genetic material,
we consider the displacment to a uniformly randomly chosen parent.)
Concretely, we may model the spatial displacement between parent and offspring
as a random quantity with some distribution;
here, we aim to infer its mean and (co)variance,
while allowing these to vary with spatial position.
For instance, pollen movement may tend to be biased downwind,
and have a higher variance in the direction of the prevailing wind.
Offspring may preferentially move towards better habitat
(or may preferentially remain there once arrived).
Certain habitat types may be more conducive to movement,
and so create ``corridors'' where thin strips of such habitat exist.




%%%%%%%%%%%%%%%%%%%%%%
\section{Methods}

\begin{itemize}
    \item Wright-Malecot as solution to 1D landscape problem
    \item talk about reverse time process
    \item segue to general PDE
    \item explain loss function, constrained optimization
    \item Lagrangian?
    \item adjoint method takes O(1) solves, direct method takes O(N) solves
    \item fenics, finite elements (arbitrary meshes, has function space framework)
    \item hot to apply to empirical data
    \item validation slimulations
    \item mesh convergence, gradient validation
\end{itemize}


%%%%%%%%%%%%%%%%%%%%%%
\section{Results}

\begin{itemize}
    \item fit to slim sims
    \item figs with synthetic and slim data fits
    \item empirical data (beetles, other?)
\end{itemize}


%%%%%%%%%%%%%%%%%%%%%%
\section{Discussion}

\begin{itemize}
    \item implications and thoughts on sampling design
    \item uncertainty
    \item dxy along genome, or locus specific
    \item interpret reverse time results to forward time
    \item non steady-state model
\end{itemize}


\bibliographystyle{unsrtnat}
\bibliography{references}


\end{document}
